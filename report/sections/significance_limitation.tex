\section{Significance}
Instead of randomly picking data points to be used in either the selection or inference stage, data fission allows both stages to take advantage of some information from each of the observations at hand. At the same time, we do not reduce the number of observations for either stages of selective inference under the data fission framework. 

\section{Limitations and challenges}

\begin{itemize}
\item paper does not discuss robustness of the method to the distribution assumptions
\item experiments do not cover cases where fissioned data get transformed to follow a different distribution
\item paper does not discuss how to choose tuning parameter (controlling amount of information split between $f(X)$ and $g(X)$)
\item following the previous point, this paper does not discuss the relations between having a discrete vs. continuous tuning parameter (e.g. the two different ways of fissioning exponentially distributed data in Appendix B of \cite{leiner2022data})
\item Proofs have missing terms, inaccurate notations.
\item The discrete version of data fission for Gamma data may be incorrect.
\end{itemize}
