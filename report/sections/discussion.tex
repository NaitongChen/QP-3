\section{Discussion}
From the simulation study in the previous section, we know that in the context of constructing selective CIs in fixed-design Gaussian linear regression models, data fission (P1) outperforms data splitting, especially in the small sample setting, in both the selection and inference stage. Even with the advantage of data fission wearing off as the sample size increases, data fission (P1) still remains competitive to data splitting. This adds to Section 4 in \cite{leiner2022data}, which demonstrates that data fission (P1) can handle high leverage points better than data splitting does in the linear regression case. Therefore, if the variance of the error term ($\sigma^2$) is known, in the context of constructing selective CIs in fixed-design Gaussian linear regression models, data fission (P1) does address at least two of the shortcomings of data splitting. It is worth noting, however, that in our simulation study, we do not check the effect of changing (1) the data splitting ratio $a$, (2) the data fission tuning parameter $\tau$, or (3) the variance of the error term $\sigma^2$. It is important to explore the effects these three parameters have on the performance of selective inference in the given context in order to get a more comprehensive understanding of the pros and cons of data fission compared to data splitting.

On the other hand, the simulation study reveals a potentially crucial problem with the data fission (P2) procedure. Namely, by adopting the data fission (P2) procedure, we no longer target the idea parameter for inference in the linear regression case. As we have observed, this leads to a notable increase in the FCR to be much higher than the target level. This raises a more general question on whether and how we can modify our corresponding selection and inference algorithm in order to accommodate cases where our inference target is different from the ideal target parameter under the data fission framework.

Finally, as briefly mentioned in \cref{sec:challenge}, the general ``conjugate prior reversal'' framework for deriving data fission procedures for distributions in the exponential family also sometimes poses challenges with regard to optimization when computing the estimated parameters in the context of constructing selective CIs in fixed-design generalized linear models. While \cite{leiner2022data} develops working solutions for when the estimated parameters are difficult to compute by changing the underlying likelihood functions, the robustness of this working solution to the modified likelihood functions remains unclear.

To summarize, there are still many important questions that needs to be investigated with respect to the general ``conjugate prior reversal'' framework for deriving data fission procedures for distributions in the exponential family. Given the generality and flexibility of the data fission procedure, it is of great interest to answer these questions so that we can provide guidance when using this method in practice. Once these questions are addressed, data fission can become a preferable alternative to data splitting and data carving due to its improved performance on selective inference, as well as its wider applicability in terms of the choice of selection and inference algorithms.