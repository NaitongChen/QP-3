\section{Proofs}
\bexa
Data fission can be viewed as a continuous analog of data splitting in terms of the allocation of Fisher information.
\newline $ $
Let $\{X_i\}_{i=1}^n$ be \iid $\distNorm(\theta, \sigma^2)$. Let $X \defas \begin{bmatrix} X_1, \dots, X_n \end{bmatrix}^\top$. Recall that data splitting defines $f(X)$ and $g(X)$ as
\[
f^{\text{split}}(X) = \begin{bmatrix} X_1, \dots, X_{an} \end{bmatrix},\quad
g^{\text{split}}(X) = \begin{bmatrix} X_{an+1}, \dots, X_{n} \end{bmatrix},
\]
Note that
\[
\mcI_{f^{\text{split}}(X)} = an\frac{1}{\sigma^2},\quad
\mcI_{g^{\text{split}}(X)} = (1-a)n\frac{1}{\sigma^2}.
\]
On the other hand, data fission first simulates $\{Z_i\}_{i=1}^n$ distributed as \iid $\distNorm(0, \sigma^2)$ and have, for some fixed $\tau\in(0,\infty)$,
\[
f^{\text{fisson}}(X) = \begin{bmatrix} X_1 + \tau Z_1, \dots, X_n + \tau Z_n \end{bmatrix},\quad
g^{\text{fisson}}(X) = \begin{bmatrix} X_1 - \frac{1}{\tau}Z_1, \dots, X_n - \frac{1}{\tau}Z_n \end{bmatrix}.
\]
Note that for all $i\in\{1,\dots,n\}$, $X_i + \tau Z_i \distas \distNorm\left(\theta, (1+\tau^2)\sigma^2 \right)$, $X_i - \frac{1}{\tau} Z_i \distas \distNorm\left(\theta, (1+\frac{1}{\tau^2})\sigma^2 \right)$. We then have
\[
\mcI_{f^{\text{fisson}}(X)} = n\frac{1}{(1+\tau^2)\sigma^2},\quad
\mcI_{g^{\text{fisson}}(X)} = n\frac{1}{(1+\frac{1}{\tau^2})\sigma^2}.
\]
By setting $a = \frac{1}{1+\tau^2}$, we have $\mcI_{f^{\text{split}}(X)} = \mcI_{f^{\text{fisson}}(X)}$ and $\mcI_{g^{\text{split}}(X)} = \mcI_{g^{\text{fisson}}(X)}$.
\eexa

\bthm\label{thm:conjugacy}
Suppose that for some $A(\cdot), \phi(\cdot), m(\cdot), \theta_1, \theta_2, H(\cdot, \cdot)$, the density of $X$ is given by
\[
p(x \given \theta_1, \theta_2) = m(x)H(\theta_1, \theta_2)\exp\{ \theta_1^\top \phi(x) - \theta_2^\top A(\phi(x)) \}.
\]
Suppose also that there exists $h(\cdot), T(\cdot), \theta_3$ such that
\[
p(z \given x, \theta_3) = h(z)\exp\{ \phi(x)^\top T(z) - \theta_3^\top A(\phi(x)) \}
\]
is a well-defined distribution. First, draw $Z \distas p(z \given X, \theta_3)$, and let $f(X) \defas Z$ and $g(X) \defas X$. Then, $(f(X), g(X))$ satisfy the data fission property (P2). Specifically, note that $f(X)$ has a known marginal distribution
\[
p(z\given \theta_1, \theta_2, \theta_3) = h(z) \frac{H(\theta_1, \theta_2)}{H(\theta_1 + T(z), \theta_2 + \theta_3)},
\]
while $g(X)$ has a known conditional distribution given $f(X)$, which is
\[
p(x\given z, \theta_1, \theta_2, \theta_3) = p(x \given \theta_1 + T(z), \theta_2 + \theta_3).
\]
\ethm
\bprf
Note that because the density $p(z \given x, \theta_3)$ must integrate to $1$, we can view the function $H(\theta_1, \theta_2)$ as a normalization factor since
\[
H(\theta_1, \theta_2) = \frac{1}{\int_{-\infty}^\infty m(x)\exp\{ \theta_1^\top \phi(x) - \theta_2^\top A(\phi(x)) \} dx}.
\]
Therefore, to compute the marginal density, we have
\[
p(z \given \theta_1, \theta_2, \theta_3) &= 
\int_{-\infty}^\infty m(x)h(z)H(\theta_1, \theta_2)\exp \{ (T(z) + \theta_1)^\top \phi(x) - (\theta_2 + \theta_3)^\top A(\phi(x)) \} dx\\
&= h(z)\frac{H(\theta_1, \theta_2)}{H(\theta_1 + T(z), \theta_2 + \theta_3)}.
\]
Similarly, the computation of the conditional density is straightforward
\[
p(x \given z, \theta_1, \theta_2, \theta_3) &= \frac{m(x)h(z)H(\theta_1, \theta_2)\exp\{(T(z)+\theta_1)^\top \phi(x) - (\theta_2 + \theta_3)^\top A(\phi(x))\}}{h(z)\frac{H(\theta_1, \theta_2)}{H(\theta_1 + T(z), \theta_2+\theta_3)}}\\
&= m(x)H(\theta_1 + T(z), \theta_2 + \theta_3)\exp\{ \phi(x)^\top (\theta_1 + T(z)) - (\theta_2 + \theta_3)^\top A(\phi(x)) \}\\
&= p(x \given \theta_1 + T(z), \theta_2 + \theta_3).
\]
This completes the proof.
\eprf

\bexa
Suppose $X\distas \distGam(\alpha, \beta)$. Draw $Z=(Z_1,\cdots,Z_B)$ where each element is \iid $Z_i \distas \distPoiss(X)$ and $B\in\{1,2,\dots\}$ is a tuning parameter. Let $f(X) = Z$ and $g(X) = X$.
\newline $ $
By writing
\[
\distGam( x \given \alpha, \beta) = \frac{\beta^\alpha}{\Gamma(\alpha)}\exp\{(\alpha-1)\log x - \beta x \},
\]
where $\distGam(\cdot \given \alpha, \beta)$ denotes the pdf of $\distGam(\alpha, \beta)$, we have that $\theta_2 = \beta, \theta_1 = \alpha-1, \phi(x) = \log x, A(\phi(x)) = \exp(\phi(x)) = \exp(\log x) = x, m(x)=1$. Therefore, $H(\theta_1,\theta_2) = \frac{\theta_2^{(\theta_1+1)}}{\Gamma(\theta_1+1)}$. Now since
\[
\distPoiss(z \given x) &= \prod_{i=1}^B\frac{1}{z_i!}\exp\{z_i \log x - x\}\\
&= \left(\prod_{i=1}^B\frac{1}{z_i!}\right)\exp\left\{ \left(\sum_{i=1}^B z_i\right) \log x - Bx \right\},
\]
we have that $h(z) = \prod_{i=1}^B \frac{1}{z_i!}, T(z) = \sum_{i=1}^B z_i, \theta_3 = B$. Therefore, by \cref{thm:conjugacy}, when $B=1$,
\[
p(z \given \theta_1,\theta_2,\theta_3) &= \frac{1}{z!}\frac{\frac{\beta^{\alpha}}{\Gamma(\alpha)}}{\frac{(\beta+1)^{\left(\alpha + z\right)}}{\Gamma\left( \alpha + z \right)}}\\
&= \frac{(\alpha+z-1)!}{(\alpha-1)!z!}\left(\frac{\beta}{\beta+1}\right)^\alpha\left(\frac{1}{\beta+1}\right)^z\\
&= \distNB\left(z\given \alpha, \frac{\beta}{\beta+1} \right);\\
p(x \given z, \theta_1, \theta_2, \theta_3) 
&= \frac{(\beta+1)^{\left( \alpha+z \right)}}{\Gamma\left( \alpha + z \right)}
\exp\left\{ \left( \alpha+z-1 \right)\log(x) - (\beta+1)x \right\}\\
&= \distGam\left(x \given \alpha+z, \beta+1 \right).
\]
However, when $B>1$,
\[
p(z \given \theta_1,\theta_2,\theta_3) &= \left(\prod_{i=1}^B \frac{1}{z_i!}\right)\frac{\frac{\beta^{\alpha}}{\Gamma(\alpha)}}{\frac{(\beta+B)^{\left(\alpha + \sum_{i=1}^B z_i\right)}}{\Gamma\left( \alpha + \sum_{i=1}^B z_i \right)}}\\
&\neq \prod_{i=1}^B \distNB\left( z_i \given \alpha, \frac{\beta}{\beta+1} \right);\\
p(x \given z, \theta_1, \theta_2, \theta_3) 
&= \frac{(\beta+B)^{\left( \alpha+\sum_{i=1}^B z_i \right)}}{\Gamma\left( \alpha + \sum_{i=1}^B z_i \right)}
\exp\left\{ \left( \alpha-1+\sum_{i=1}^B z_i \right)\log(x) - (\beta+B)x \right\}\\
&= \distGam\left(x \given \alpha+\sum_{i=1}^B z_i, \beta+B \right).
\]
\eexa

\bexa
Suppose $X\distas \distGam(\alpha, \beta)$. Draw $Z \distas \distPoiss(\tau X)$, where $\tau\in(0,\infty)$ is a tuning parameter. Let $f(X) = Z$ and $g(X) = X$.
\newline $ $
By writing
\[
\distGam( x \given \alpha, \beta) = \frac{\beta^\alpha}{\Gamma(\alpha)}\exp\{(\alpha-1)\log x - \beta x \},
\]
where $\distGam(\cdot \given \alpha, \beta)$ denotes the pdf of $\distGam(\alpha, \beta)$, we have that $\theta_2 = \beta, \theta_1 = \alpha-1, \phi(x) = \log x, A(\phi(x)) = \exp(\phi(x)) = \exp(\log x) = x, m(x)=1$. Therefore, $H(\theta_1,\theta_2) = \frac{\theta_2^{(\theta_1+1)}}{\Gamma(\theta_1+1)}$. Now since
\[
\distPoiss(z \given \tau x) &= \frac{1}{z!}\exp\{z \log (\tau x) - \tau x\}\\
&= \frac{1}{z!}\exp\{z \log \tau + z \log x - \tau x\}\\
&= \frac{\tau^z}{z!}\exp\{z \log x - \tau x\},
\]
we have that $h(z) = \frac{\tau^z}{z!}, T(z) = z, \theta_3 = \tau$. Therefore, by \cref{thm:conjugacy},
\[
p(z \given \theta_1,\theta_2,\theta_3) &= \frac{\tau^z}{z!}\frac{\frac{\beta^{\alpha}}{\Gamma(\alpha)}}{\frac{(\beta+\tau)^{\left(\alpha + z\right)}}{\Gamma\left( \alpha + z \right)}}\\
&= \frac{(\alpha+z-1)!}{(\alpha-1)!z!}\left(\frac{\beta}{\beta+\tau}\right)^\alpha\left(\frac{\tau}{\beta+\tau}\right)^z\\
&= \distNB\left(z\given \alpha, \frac{\beta}{\beta+\tau} \right);\\
p(x \given z, \theta_1, \theta_2, \theta_3) 
&= \frac{(\beta+\tau)^{\left( \alpha+z \right)}}{\Gamma\left( \alpha + z \right)}
\exp\left\{ \left( \alpha+z-1 \right)\log(x) - (\beta+\tau)x \right\}\\
&= \distGam\left(x \given \alpha+z, \beta+\tau \right).
\]
\eexa